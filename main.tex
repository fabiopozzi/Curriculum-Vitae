%%%%%%%%%%%%%%%%%%%%%%%%%%%%%%%%%%%%%%%
% Deedy - One Page Two Column Resume
% LaTeX Template
% Version 1.1 (30/4/2014)
%
% Original author:
% Debarghya Das (http://debarghyadas.com)
%
% Original repository:
% https://github.com/deedydas/Deedy-Resume
%
% IMPORTANT: THIS TEMPLATE NEEDS TO BE COMPILED WITH XeLaTeX
%
% This template uses several fonts not included with Windows/Linux by
% default. If you get compilation errors saying a font is missing, find the line
% on which the font is used and either change it to a font included with your
% operating system or comment the line out to use the default font.
% 
%%%%%%%%%%%%%%%%%%%%%%%%%%%%%%%%%%%%%%
% 
% TODO:
% 1. Integrate biber/bibtex for article citation under publications.
% 2. Figure out a smoother way for the document to flow onto the next page.
% 3. Add styling information for a "Projects/Hacks" section.
% 4. Add location/address information
% 5. Merge OpenFont and MacFonts as a single sty with options.
% 
%%%%%%%%%%%%%%%%%%%%%%%%%%%%%%%%%%%%%%
%
% CHANGELOG:
% v1.1:
% 1. Fixed several compilation bugs with \renewcommand
% 2. Got Open-source fonts (Windows/Linux support)
% 3. Added Last Updated
% 4. Move Title styling into .sty
% 5. Commented .sty file.
%
%%%%%%%%%%%%%%%%%%%%%%%%%%%%%%%%%%%%%%%
%
% Known Issues:
% 1. Overflows onto second page if any column's contents are more than the
% vertical limit
% 2. Hacky space on the first bullet point on the second column.
%
%%%%%%%%%%%%%%%%%%%%%%%%%%%%%%%%%%%%%%

\documentclass[]{deedy-resume-openfont}

\usepackage{etoolbox}
\AtBeginEnvironment{tightemize}{\itshape\small}
\begin{document}

%%%%%%%%%%%%%%%%%%%%%%%%%%%%%%%%%%%%%%
%
%     TITLE NAME
%
%%%%%%%%%%%%%%%%%%%%%%%%%%%%%%%%%%%%%%
\namesection{Fabio}{Pozzi}{
\href{mailto:pozzi.fabio@gmail.com}{pozzi.fabio@gmail.com}\\
333 3005059
}

\begin{minipage}[t]{0.90\textwidth} 

%%%%%%%%%%%%%%%%%%%%%%%%%%%%%%%%%%%%%%
%     EXPERIENCE
%%%%%%%%%%%%%%%%%%%%%%%%%%%%%%%%%%%%%%

\section{Sommario}
\vspace*{2mm}
Attualmente insegno informatica presso il Liceo Scientifico San Lorenzo.\\
Mi sono occupato di sviluppo software e firmware nell’ambito della gestione di sensori e dispositivi embedded interconnessi tramite diverse
tecnologie radio (WiFi, Bluetooth, Sigfox).\\
Ho maturato una significativa esperienza come sviluppatore software nel settore automotive motorsport. Ho infatti partecipato allo sviluppo di più
progetti, a partire dal sistema di telemetria dati ed audio per i campionati di Formula 1, WEC e Formula-E.\\
Durante la mia carriera lavorativa ho maturato anche esperienza nello sviluppo di applicazioni web, principalmente con l’uso del framework Ruby
On Rails.\\
Ho una buona conoscenza dei linguaggi Python, Go e C++.\\
Ho esperienza nell’uso di SVN e git per il versioning del codice.
\vspace*{2mm}
%------------------------------------------------
% Experience
%------------------------------------------------

\section{Esperienza}
\vspace*{2mm}

\runsubsection{Liceo Scientifico San Lorenzo}
\descript{| Insegnante informatica}

\location{Settembre 2018 – presente | Novara (NO)}
Docenza di informatica in cinque classi della sezione scienze applicate del liceo scientifico.\\
Docente del corso di preparazione all'esame ECDL offerto dalla
scuola.\\
Responsabile per il cyberbullismo dell'istituto.
\vspace{\topsep}

\runsubsection{Reios srl}
\descript{| Sviluppatore embedded}

\location{Giugno 2017 – Dicembre 2017 | Novara (NO)}
Sviluppo e valutazione performance su piattaforma Linux ed STM32 di un sistema di comunicazione basato su Wi-Fi mesh.
Sviluppo firmware acquisizione e controllo per applicazioni IoT.
\vspace{\topsep} % Hacky fix for awkward extra vertical space

\runsubsection{FIT2YOU Broker}
\descript{| Sviluppatore Ruby on Rails}

\location{Gennaio 2017 – Giugno 2017 | Vigevano (PV)}
Sviluppo web di una piattaforma B2B di supporto alla vendita di polizze assicurative auto.
\vspace{\topsep} % Hacky fix for awkward extra vertical space

\runsubsection{Magneti Marelli Motorsport}
\descript{| Sviluppatore embedded}

\location{Dicembre 2011 – Dicembre 2016 | Corbetta (MI)}
Sviluppo software di gestione flussi telemetria audio e dati lato macchina:
\vspace{\topsep} % Hacky fix for awkward extra vertical space
\begin{tightemize}
\item Sviluppo di software applicativo in C
\item Sviluppo interfaccia web e backend per gestione centralina telemetria
\item Sviluppo firmware su piattaforma PowerPC
\item Sviluppo software validazione e test hardware
\item Sviluppo Linux kernel drivers
\end{tightemize}
\vspace*{4mm}
%------------------------------------------------
\runsubsection{Ennova research}
\descript{| Sviluppatore web}

\location{Settembre 2011 – Dicembre 2011 | Lomazzo (CO)}
Sviluppo applicazione web basata su Ruby on Rails per la pianificazione e la gestione della distribuzione di riviste.
\vspace{\topsep} % Hacky fix for awkward extra vertical space
\begin{tightemize}
\item Sviluppo backend
\item Gestione interazione con applicazione mobile Android
\item Gestione pianificazione con uso API Google maps
\end{tightemize}
\vspace*{4mm}
%------------------------------------------------
\runsubsection{Dsmart s.r.l.}
\descript{| Sviluppatore web}

\location{Ottobre 2010 – Aprile 2011 | Bussero (MI)}
Sviluppo applicazioni web basate su Ruby on Rails per la gestione di associazioni sportive.
\vspace*{4mm}
%------------------------------------------------
%\sectionspace % Some whitespace after the section
\end{minipage}
\newpage % Start a new page
\begin{minipage}[t]{0.90\textwidth} % The right column takes up 66% of the text width of the page
%------------------------------------------------
% Education
%------------------------------------------------

\section{Formazione} 
\vspace*{4mm}
\subsection{Università Bicocca di Milano}

\descript{24cfu}
\vspace*{1mm}
Ho acquisito i 24cfu in ambito antropo-psico-pedagogico, metodologie e tecnologie didattiche.

\vspace*{4mm}
\subsection{Politecnico di Milano}

\descript{Laurea specialistica in Ingegneria Informatica}
\location{05/2008 - 12/2011 \\ Voto: 101/110}
\vspace*{1mm}
Analisi, progettazione e sviluppo di contromisure per vulnerabilità basate su alterazioni del flusso di controllo.\\
LibDefender: una libreria dinamica per garantire l'integrità del flusso di esecuzione.

\vspace*{4mm}

\descript{Laurea triennale in Ingegneria Informatica}
\location{09/2003 - 05/2008 \\ Voto: 99/110}
\vspace*{1mm}
Implementazione di un sistema di intrusion prevention nel kernel Linux.
\vspace*{10mm}

%------------------------------------------------
% Research
%------------------------------------------------

\section{Pubblicazioni}
\vspace*{3mm}
\runsubsection{Drop-In Control Flow Hijacking Prevention through Dynamic Library Interception}
\descript{ \\ Information Technology: New Generations (ITNG)}
\vspace*{1mm}
\location{Aprile 2013 | Las Vegas, NV}
 % Some whitespace after the section
\vspace*{10mm}
%----------------------------------------------------------------------------------------


%------------------------------------------------
% Skills
%------------------------------------------------
\section{Competenze}
\descript{Linguaggi di programmazione}
\vspace*{4mm}
\begin{tightemize}
  \item C
  \item Ruby
  \item Python 
  \item Go
\end{tightemize}
\vspace*{4mm}
\descript{Version Control Systems}
\vspace*{4mm}
\begin{tightemize}
  \item Git
  \item Subversion
\end{tightemize}

\vspace*{10mm}

\section{Lingue}
\vspace*{4mm}
Inglese \\ TOEFL 97/110 (2009) 


\vspace*{12mm}
Autorizzo il trattamento dei miei dati personali ai sensi dell’art. 13 d. lgs. 30 giugno 2003 n°196 \\ “Codice in materia di protezione dei dati personali” e dell’art. 13 GDPR 679/16 \\ “Regolamento europeo sulla protezione dei dati personali”.

%----------------------------------------------------------------------------------------

\hfill
\end{minipage} % The end of the right column

\end{document}  \documentclass[]{article}